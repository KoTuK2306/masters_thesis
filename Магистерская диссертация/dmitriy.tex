\documentclass[a4paper]{report}

\def\baselinestretch{1.1}
\usepackage[14pt]{extsizes}
\usepackage[utf8]{inputenc}
\usepackage[russian]{babel}
\usepackage{indentfirst}
\usepackage{mathrsfs}

%%%%%%%%%%%%%%%%% Символы, графика %%%%%%%%%%%%%%%%%%%%%

\usepackage[T2A]{fontenc}
\usepackage{amsmath,amssymb,amsfonts,amsthm}
\newcommand\dsone{\mathds{H}}
\usepackage{graphicx}
\usepackage{color}
\usepackage[pdftex,colorlinks,linkcolor=blue,citecolor=blue]{hyperref}
\usepackage{pgfplots}
\usepackage{tikz}
\usepackage{array}
\newcolumntype{P}[1]{>{\centering\arraybackslash}p{#1}}

%%%%%%%%% Разметка страницы %%%%%%%%%

\bibliographystyle{plain}  % Change this to your preferred style
\renewcommand{\thetable}{\arabic{table}}
\usepackage{indentfirst}
\topmargin=-1.5cm %отступ сверху
\oddsidemargin=0.4cm %отступ слева (нечетные страницы)
\evensidemargin=0.4cm %(четные страницы)
\textwidth=16cm %ширина текста
\textheight=24cm
\tolerance=800
\parskip=1ex

\pagestyle{plain}

\usepackage{listings}

\definecolor{codegreen}{rgb}{0,0.6,0}
\definecolor{codegray}{rgb}{0.5,0.5,0.5}
\definecolor{codepurple}{rgb}{0.58,0,0.82}
\definecolor{backcolour}{rgb}{0.95,0.95,0.92}

\lstset{
    extendedchars=true,  % Corrected this line
    backgroundcolor=\color{backcolour},
    commentstyle=\color{codegreen},
    keywordstyle=\color{magenta},
    numberstyle=\tiny\color{codegray},
    stringstyle=\color{codepurple},
    breakatwhitespace=false,
    breaklines=true,
    captionpos=b,
    keepspaces=true,
    numbers=left,
    numbersep=3pt,
    showspaces=false,
    showstringspaces=false,
    showtabs=false,
    tabsize=1,
    basicstyle=\fontsize{10}{12}\selectfont\ttfamily
}

\newcommand{\ket}[1] {{\ensuremath{\left|#1\right\rangle}}}
\newcommand{\bra}[1] {{\ensuremath{\left\langle#1\right|}}}
\newcommand{\ketbra}[2]{{\ensuremath {\left|#1\right\rangle\!\;\!\!\left\langle#2\right|}}}
\newcommand{\braket}[2]{{\ensuremath {\left\langle#1\left|\!\right.#2\right\rangle}}}

\begin{document}

\begin{titlepage}
	\begin{center}
		Министерство науки и высшего образования РФ\\
		ФГБОУ ВО «Тверской государственный университет»\\
		Математический факультет\\
		Направление 02.04.01 Математика и компьютерные науки\\
		Профиль <<Математическое и компьютерное моделирование>>	
	\end{center}
	
	\vspace{1.4cm}
	\begin{center}
	
		{МАГИСТЕРСКАЯ ДИССЕРТАЦИЯ}	
		
		\vspace{1.0cm}
    \large{Вариационный квантовый алгоритм с оптимизацией методом отжига}
		
		
		\vspace{1.0cm}
	\end{center}
	
	
	
	\begin{flushright}
		\begin{minipage}{80mm}
			Автор:\\
			Алешин Д.А.\\
      Подпись:
			
			\vspace{1.0cm}
			Научный руководитель:\\
			д. ф.-м. н. Цирулёв А.Н.\\
      Подпись:
			
		\end{minipage}
	\end{flushright}
	
	
	\vspace{1.6cm}
	\noindent Допущен к защите:\\
	Руководитель ООП: Цветков В.П.\\[0.3cm]
  $\underset{\textit{(подпись, дата)}}{\underline{\hspace{0.3\textwidth}}}$
	\vspace{2.2cm}
	
	
	
	\begin{center}
		Тверь 2025
	\end{center}
	
	\date{}
\end{titlepage}

\setcounter{page}{2}

\tableofcontents
\newpage

% Abstract

%#########################################################################################################################################################################################################################
%##################################################################################################### Введение ##########################################################################################################
%#########################################################################################################################################################################################################################

\addcontentsline{toc}{chapter}{\hspace{5.5mm} Введение}
\chapter*{Введение}

%#########################################################################################################################################################################################################################
%############################################################## 1 Вариационные квантовые алгоритмы: общая схема ##########################################################################################################
%#########################################################################################################################################################################################################################

\chapter{Вариационные квантовые алгоритмы: общая схема}

%###################################################################################################################################################################################################################################
%######################################################################################################### 1.1 Базис Паули #########################################################################################################
%###################################################################################################################################################################################################################################

\section{Базис Паули}
%###################################################################################################################################################################################################################################
%######################################################################################################### 1.1.1 Матричная форма ###################################################################################################
%###################################################################################################################################################################################################################################

\subsection{Матричная форма}

Матрицы Паули, обозначаемые как $\sigma_x$, $\sigma_y$, и $\sigma_z$, представляют собой набор эрмитовых и унитарных $2 \times 2$ матриц. Эти матрицы играют центральную роль в описании квантовых систем с полуцелым спином, таких как электроны, и занимают важное место в теории представлений группы SU(2). Их использование охватывает широкий спектр задач в квантовой механике и квантовой теории поля.

Матрицы Паули определяются следующим образом:

$$
\sigma_x = \begin{pmatrix}
0 & 1 \\
1 & 0
\end{pmatrix}, \quad
\sigma_y = \begin{pmatrix}
0 & -i \\
i & 0
\end{pmatrix}, \quad
\sigma_z = \begin{pmatrix}
1 & 0 \\
0 & -1
\end{pmatrix}.
$$


Эти матрицы имеют ряд уникальных свойств, включая эрмитовость ($\sigma_i = \sigma_i^\dagger$) и унитарность ($\sigma_i \sigma_i^\dagger = I$). Они также удовлетворяют специфическим соотношениям коммутации и антикоммутации, что делает их полезными для описания квантовых преобразований и взаимодействий.

Коммутационные соотношения для матриц Паули выражаются следующим образом:

$$
[\sigma_i, \sigma_j] = 2i\epsilon_{ijk}\sigma_k,
$$

где $\epsilon_{ijk}$ — символ Леви-Чивиты. Эти соотношения играют важную роль в понимании квантовой динамики и описании вращений в спиновых системах.

Антикоммутационные соотношения для матриц Паули выглядят следующим образом:

$$
\{\sigma_i, \sigma_j\} = 2\delta_{ij}I,
$$

где $\delta_{ij}$ — символ Кронекера, а $I$ — единичная матрица. Эти свойства широко используются в квантовых вычислениях и других приложениях, где важна как коммутация, так и антикоммутация операторов.

Матрицы Паули находят применение в описании операторов спина. Оператор спина частицы можно выразить через линейную комбинацию матриц Паули:

$$
\vec{S} = \frac{\hbar}{2} \vec{\sigma},
$$

где $\vec{S}$ — оператор спина, $\hbar$ — приведенная постоянная Планка. Это позволяет моделировать взаимодействие спина с внешними полями и другими квантовыми системами.

В квантовой информации матрицы Паули формируют базис для всех эрмитовых матриц размерности $2 \times 2$. Любой эрмитов оператор может быть представлен как:

$$
A = a_0 I + a_x \sigma_x + a_y \sigma_y + a_z \sigma_z,
$$

где $a_0, a_x, a_y, a_z$ — вещественные коэффициенты. Это представление используется в квантовой томографии и других вычислительных задачах.

Дополнительно, матрицы Паули являются генераторами группы вращений $SU(2)$, что делает их важными в теории спиновых вращений. Они используются для описания квантовых систем, таких как атомы, молекулы и твёрдые тела, где спин играет ключевую роль. Эти матрицы помогают в разработке теоретических моделей, которые описывают явления на микроуровне, такие как квантовые переходы и спиновые волны.

Использование матриц Паули в квантовых вычислениях также позволяет эффективно кодировать и манипулировать квантовой информацией. Они применяются в реализации квантовых гейтов, таких как X-гейт (физически соответствует $\sigma_x$), Y-гейт и Z-гейт, которые являются базовыми элементами в построении квантовых алгоритмов.

%#####################################################################################################################################################################################################################################
%######################################################################################################### 1.1.2 Операторная форма ###################################################################################################
%#####################################################################################################################################################################################################################################

\subsection{Операторная форма}

Перейдем к рассмотрению операторной формы матриц Паули в контексте квантовой системы из $n$ кубитов. Каждый кубит связан с двумерным Гильбертовым пространством $\mathcal{H}$. Для системы из $n$ кубитов, полное пространство $\mathcal{H}_n$ образуется как тензорное произведение индивидуальных пространств, и его размерность равна $2^n$.

Теперь определим операторную форму матриц Паули в этом пространстве. Для ортонормированного базиса $\{|0\rangle, |1\rangle\}$, соответствующего каждому кубиту, мы можем выразить операторы Паули как:
\begin{align*}
    \hat{\sigma}_0 = |0\rangle\langle0| + |1\rangle\langle1|,\; \hat{\sigma}_1 = |0\rangle\langle1| + |1\rangle\langle0|,     \\
    \hat{\sigma}_2 = -i|0\rangle\langle1| + i|1\rangle\langle0|,\;  \hat{\sigma}_3 = |0\rangle\langle0| - |1\rangle\langle1|,
\end{align*}

Эти операторы эрмитовы и унитарны и составляют основу в $\mathcal{H}$. Каждый из них может быть выражен в виде тензорного произведения операторов на отдельных кубитах:
$$
\hat{\sigma}_{k_1\ldots k_n} = \hat{\sigma}_{k_1} \otimes \ldots \otimes \hat{\sigma}_{k_n},
$$
где $k_i \in \{0, 1, 2, 3\}$. Таким образом, операторная форма матриц Паули позволяет описывать сложные квантовые состояния и преобразования в системах из нескольких кубитов.

Дополнительно, в многокубитных системах операторы Паули используются для построения сложных квантовых операторов, таких как гамильтонианы взаимодействия. Например, гамильтониан Изинга, используемый для моделирования спиновых систем, может быть выражен через сумму операторов Паули:
$$
H = \sum_{i=1}^{n} J_i \hat{\sigma}_z^{(i)} + \sum_{i<j} J_{ij} \hat{\sigma}_z^{(i)} \hat{\sigma}_z^{(j)},
$$
где $J_i$ и $J_{ij}$ — коэффициенты, описывающие взаимодействия между спинами.

Операторная форма матриц Паули также играет ключевую роль в квантовой теории измерений. Операторы проектора, используемые для моделирования квантовых измерений, могут быть представлены через матрицы Паули, что позволяет эффективно описывать процессы детекции квантовых состояний.

%#########################################################################################################################################################################################################################################
%######################################################################################################### 1.2 Целевая функция и анзац ###################################################################################################
%#########################################################################################################################################################################################################################################

\section{Целевая функция и анзац}

Целевая функция и анзац являются ключевыми компонентами вариационного квантового алгоритма (VQA). Целевая функция служит критерием оптимизации, тогда как анзац представляет собой параметризованное квантовое состояние, которое подлежит оптимизации. В этом разделе мы подробно рассмотрим теоретические аспекты этих компонентов и их значение в контексте VQA.

Целевая функция, в контексте квантовых вычислений, часто представляется в виде гамильтониана, который можно записать в базисе Паули. Пусть $H$ — это гамильтониан, описывающий систему, и он может быть выражен как линейная комбинация операторов Паули:

\begin{equation}
H = \sum_i c_i P_i,
\end{equation}

где $c_i$ — это вещественные коэффициенты, а $P_i$ — тензорные произведения матриц Паули, например, $I$, $X$, $Y$, $Z$. Цель вариационного алгоритма — минимизировать ожидаемое значение этого гамильтониана в состоянии $\ket{\psi(\vec{\theta})}$:

\begin{equation}
E(\vec{\theta}) = \bra{\psi(\vec{\theta})} H \ket{\psi(\vec{\theta})}.
\end{equation}

Анзац $\ket{\psi(\vec{\theta})}$ — это параметризованное квантовое состояние, которое можно представить как последовательность квантовых гейтов, зависящих от набора параметров $\vec{\theta}$. В общем случае, анзац представляется следующим образом:

\begin{equation}
\ket{\psi(\vec{\theta})} = U(\vec{\theta}) \ket{\phi_0},
\end{equation}

где $U(\vec{\theta})$ — это унитарный оператор, зависящий от параметров $\vec{\theta}$, и $\ket{\phi_0}$ — начальное состояние квантовой системы, часто принимаемое за простое состояние, например, $\ket{0}^{\otimes n}$ для $n$ кубитов.

Выбор анзаца критически важен для эффективности алгоритма. Он должен быть достаточно выразительным, чтобы покрывать широкий спектр возможных квантовых состояний, и в то же время достаточно простым для реализации на квантовом компьютере. Например, часто используются такие анзацы, как Hardware Efficient Ansatz, который состоит из чередования локальных одно- и двухкубитных гейтов, и анзац, основанный на теории матричных произведений (MPS).

Оптимизация параметров $\vec{\theta}$ сводится к минимизации целевой функции $E(\vec{\theta})$. Для этого могут быть использованы различные классические оптимизационные методы, такие как градиентные методы, методы стохастического градиентного спуска или эволюционные алгоритмы. Важно отметить, что вычисление градиента целевой функции может быть выполнено на квантовом компьютере с помощью метода параметрического сдвига:

\begin{equation}
\frac{\partial E(\vec{\theta})}{\partial \theta_i} = \frac{1}{2} \left( E(\vec{\theta} + \frac{\pi}{2} \vec{e_i}) - E(\vec{\theta} - \frac{\pi}{2} \vec{e_i}) \right),
\end{equation}

где $\vec{e_i}$ — это единичный вектор, соответствующий параметру $\theta_i$.

Продолжая рассмотрение, целевая функция и анзац в вариационных квантовых алгоритмах являются неотъемлемыми элементами, определяющими успех алгоритма. Гамильтониан, использующийся в целевой функции, может быть более сложным, включающим как одночастичные, так и многочастичные взаимодействия. Его представление в базисе Паули позволяет эффективно преобразовывать квантовые гейты в аппаратные реализации на квантовом процессоре. Рассмотрим более сложное представление гамильтониана:

\begin{equation}
H = \sum_{i} h_i P_i + \sum_{i < j} h_{ij} P_i P_j,
\end{equation}

где $h_i$ и $h_{ij}$ — это коэффициенты взаимодействий, а $P_i$ и $P_j$ — операторы Паули.

Анзац, как выраженное состояние, должен быть устойчивым к аппаратным ошибкам и одновременно выразительным. Например, анзац UCCSD (Unitary Coupled Cluster Singles and Doubles) представляет собой:

\begin{equation}
\ket{\psi(\vec{\theta})} = e^{T(\vec{\theta}) - T^\dagger(\vec{\theta})} \ket{\phi_0},
\end{equation}

где $T(\vec{\theta}) = \sum_i \theta_i \tau_i$ — кластерный оператор, а $\tau_i$ — генераторы одно- и двухчастичных возмущений.

Выбор анзаца играет критическую роль в симуляции корреляций в квантовых системах. Например, анзацы, такие как ADAPT-VQE, динамически настраивают структуру анзаца в ходе оптимизации, что позволяет уменьшить количество параметров:

\begin{equation}
\ket{\psi(\vec{\theta})} = U_1(\theta_1) U_2(\theta_2) \dots U_n(\theta_n) \ket{\phi_0},
\end{equation}

где каждый $U_i(\theta_i)$ добавляется в процессе оптимизации.

Оптимизация параметров $\vec{\theta}$ осуществляется с использованием классических алгоритмов. Методы, такие как стохастический градиентный спуск (SGD) или метод Бройдена — Флетчера — Гольдфарба — Шанно (BFGS), часто применяются в квантовых вычислениях. Градиент целевой функции может быть вычислен с помощью метода параметрического сдвига, что позволяет эффективно управлять процессом оптимизации:

\begin{equation}
\frac{\partial E(\vec{\theta})}{\partial \theta_i} = \frac{1}{2} \left( E(\vec{\theta} + \frac{\pi}{2} \vec{e_i}) - E(\vec{\theta} - \frac{\pi}{2} \vec{e_i}) \right).
\end{equation}

Глубокое понимание целевой функции и анзаца требует изучения их математических и физических основ. Например, гамильтониан может включать взаимодействия, которые описываются не только одночастичными, но и многочастичными терминами. Это приводит к гамильтонианам вида:

\begin{equation}
H = \sum_i h_i P_i + \sum_{i<j} h_{ij} P_i P_j + \sum_{i<j<k} h_{ijk} P_i P_j P_k,
\end{equation}

где $h_{ijk}$ — коэффициенты, соответствующие трёхчастичным взаимодействиям. Такие гамильтонианы могут моделировать сложные физические системы, включая молекулярные и твердотельные структуры.

Анзац, с другой стороны, должен быть достаточно гибким, чтобы точно описывать физические состояния, и в то же время устойчивым к аппаратным ошибкам, возникающим на квантовом процессоре. Одним из подходов к созданию анзаца является использование теории матричных произведений (MPS), которая позволяет эффективно кодировать квантовые состояния с учётом корреляций между частицами:

\begin{equation}
\ket{\psi(\vec{\theta})} = \sum_{i_1, i_2, \ldots, i_n = 0, 1} A^{[1]}_{i_1} A^{[2]}_{i_2} \cdots A^{[n]}_{i_n} \ket{i_1 i_2 \cdots i_n},
\end{equation}

где $A^{[k]}_{i_k}$ — матрицы, зависящие от параметров $\vec{\theta}$.

Также важным аспектом является оценка точности и эффективности выбранного анзаца. Один из методов оценки — это использование метрики Фиделити, которая измеряет близость двух квантовых состояний:

\begin{equation}
F = |\langle \psi_{\text{exact}} | \psi(\vec{\theta}) \rangle|^2.
\end{equation}

%#######################################################################################################################################################################################################################################
%######################################################################################################### 1.3 Общая схема алгоритма ###################################################################################################
%#######################################################################################################################################################################################################################################

\section{Общая схема алгоритма}

Общая схема вариационного квантового алгоритма (VQA) объединяет элементы квантовых и классических вычислений, предлагая эффективный подход для решения задач, которые традиционно считались сложными для классических методов. В этом разделе мы подробно рассмотрим структуру VQA, его основные этапы и математическое описание, а также обсудим преимущества и вызовы, связанные с реализацией этих алгоритмов.

Вариационный квантовый алгоритм начинается с подготовки начального квантового состояния $\ket{\phi_0}$, которое часто выбирается в виде простого состояния, такого как $\ket{0}^{\otimes n}$, где $n$ — число кубитов в системе. Это состояние служит основой для дальнейшей эволюции квантовой системы. Следующим шагом является применение параметризованной унитарной операции $U(\vec{\theta})$, которая зависит от набора параметров $\vec{\theta}$, чтобы создать анзац — параметризованное квантовое состояние:

\begin{equation}
\ket{\psi(\vec{\theta})} = U(\vec{\theta}) \ket{\phi_0}.
\end{equation}

Выбор унитарного оператора $U(\vec{\theta})$ является ключевым, так как он должен быть достаточно гибким, чтобы охватывать множество возможных квантовых состояний, и в то же время достаточно простым для реализации на квантовом процессоре. Это могут быть гейты, такие как однокубитные вращения и двухкубитные CNOT-гейты, которые комбинируются для создания сложных квантовых операций.

Следующим шагом является измерение квантовой системы для получения ожидаемого значения целевой функции, которая, как правило, выражена через гамильтониан $H$:

\begin{equation}
E(\vec{\theta}) = \bra{\psi(\vec{\theta})} H \ket{\psi(\vec{\theta})}.
\end{equation}

Измерение осуществляется с использованием квантовых гейтов и операторов Паули. Это позволяет преобразовать квантовое состояние в измеримые величины, которые необходимы для дальнейшей оптимизации. Важно, чтобы измерения были точными, так как они напрямую влияют на результат оптимизации.

После измерения результаты передаются на классический компьютер, где выполняется оптимизация параметров $\vec{\theta}$ с целью минимизации $E(\vec{\theta})$. Классические алгоритмы, такие как метод градиентного спуска или более сложные методы, такие как BFGS, используются для обновления параметров. Этот процесс можно представить следующим образом:

\begin{equation}
\vec{\theta}_{t+1} = \vec{\theta}_t - \eta \nabla E(\vec{\theta}_t),
\end{equation}

где $\eta$ — скорость обучения, а $\nabla E(\vec{\theta}_t)$ — градиент целевой функции.

Этот цикл измерения и оптимизации продолжается до тех пор, пока не будет достигнута сходимость к минимальному значению целевой функции. Критерии сходимости могут включать достижение заданной точности или максимального числа итераций.

Важным аспектом VQA является его гибкость и адаптивность. Алгоритм может быть настроен для решения различных задач, включая квантовую химию, оптимизацию и машинное обучение. Это достигается за счет изменения целевой функции и анзаца в зависимости от конкретной задачи.

Другим важным фактором является устойчивость к аппаратным ошибкам. Квантовые процессоры подвержены шумам и ошибкам, что может влиять на точность вычислений. VQA имеет встроенные механизмы для учета и коррекции таких ошибок, что делает его подходящим для использования на современных квантовых устройствах.

Кроме того, VQA поддерживает гибридный подход, объединяя квантовые вычисления с классическими. Это позволяет использовать квантовые ресурсы для сложных вычислений, таких как поиск минимума в многомерных пространствах, в то время как классические алгоритмы используются для оптимизации и анализа результатов.

%#############################################################################################################################################################################################################################
%######################################################################################################### 1.4 Оптимизация ###################################################################################################
%#############################################################################################################################################################################################################################

\section{Оптимизация}

%#############################################################################################################################################################################################################################
%################################################################### 2 Вариационный квантовый алгоритм на основе метода отжига ###############################################################################################
%#############################################################################################################################################################################################################################

\chapter{Вариационный квантовый алгоритм на основе метода отжига}

%#############################################################################################################################################################################################################################
%##################################################################################### 2.1 Метод отжига ######################################################################################################################
%#############################################################################################################################################################################################################################

\section{Метод отжига}

Метод отжига, как фундаментальная концепция в решении задач глобальной оптимизации, находит широкое применение в квантовых вычислениях, особенно в контексте вариационных квантовых алгоритмов. Основная идея метода заключается в постепенном снижении "температуры" системы, чтобы достичь состояния минимальной энергии. В этом разделе подробно рассмотрим как классический, так и квантовый подходы к отжигу, их теоретические основы и практическое применение.

Классический метод отжига основывается на аналогии с физическим процессом термического отжига, при котором материал медленно охлаждается, чтобы избежать образования дефектов и достичь состояния минимальной энергии. Математическое основание метода связано с распределением Больцмана, которое описывает вероятность состояния системы при заданной температуре $T$:

\begin{equation}
p(x) = \frac{1}{Z(T)} \exp\left(-\frac{E(x)}{k_B T}\right),
\end{equation}

где $E(x)$ — энергия состояния $x$, $k_B$ — константа Больцмана, а $Z(T)$ — статистическая сумма. Процесс отжига моделирует систему, которая может переходить между состояниями $x$ и $y$ с вероятностью, зависящей от разности энергий $\Delta E = E(y) - E(x)$:

\begin{equation}
p(x \rightarrow y) = \min\left(1, \exp\left(-\frac{\Delta E}{k_B T}\right)\right).
\end{equation}

С течением времени, температура $T$ постепенно уменьшается, что приводит к уменьшению вероятности перехода в состояния с более высокой энергией, в то время как система стремится к состоянию глобального минимума энергии.

Квантовый алгоритм отжига использует преимущества квантовой механики, такие как суперпозиция и туннелирование, для более эффективного поиска глобального минимума. В отличие от классического подхода, квантовый отжиг позволяет системе преодолевать энергетические барьеры, используя когерентное туннелирование, что значительно увеличивает вероятность нахождения глобального минимума.

Квантовый отжиг моделируется с помощью временного гамильтониана, который постепенно изменяется от начального состояния к целевому:

\begin{equation}
H(t) = (1 - s(t)) H_B + s(t) H_P,
\end{equation}

где $H_B$ — начальный гамильтониан, часто представляющий собой простую задачу, такую как сумма операторов Паули $X$, а $H_P$ — проблема-специфический гамильтониан. Функция $s(t)$, изменяющаяся от 0 до 1, управляет эволюцией системы от начального состояния к состоянию минимальной энергии.

Эволюция квантовой системы описывается уравнением Шрёдингера:

\begin{equation}
i \hbar \frac{\partial}{\partial t} \ket{\psi(t)} = H(t) \ket{\psi(t)},
\end{equation}

где $\hbar$ — приведённая постоянная Планка. Это уравнение описывает, как квантовая система изменяется со временем под воздействием изменяющегося гамильтониана.

Важным аспектом квантового отжига является адъективная эволюция системы, которая позволяет системе оставаться в состоянии минимальной энергии в течение всего процесса. Это достигается за счёт медленного изменения параметра $s(t)$ в соответствии с адъективным теоремой:

\begin{equation}
\frac{ds}{dt} \ll \frac{\Delta^2}{\hbar \left\lVert \frac{dH}{ds} \right\rVert},
\end{equation}

где $\Delta$ — энергетический разрыв между основным и первым возбужденными состояниями гамильтониана.

%#########################################################################################################################################################################################################################
%##################################################################################### 2.2 Алгоритм ######################################################################################################################
%#########################################################################################################################################################################################################################

\section{Алгоритм}

%#########################################################################################################################################################################################################################
%############################################################### 2.3 Сравнительные результаты тестирования ###############################################################################################################
%#########################################################################################################################################################################################################################

\section{Сравнительные результаты тестирования}

%#########################################################################################################################################################################################################################
%################################################################################################### Заключение ##########################################################################################################
%#########################################################################################################################################################################################################################

\addcontentsline{toc}{chapter}{\hspace{5.5mm} Заключение}
\chapter*{Заключение}


%#########################################################################################################################################################################################################################
%################################################################################################### Литература ##########################################################################################################
%#########################################################################################################################################################################################################################

\addcontentsline{toc}{chapter}{\hspace{5.5mm} Литература}
\begin{thebibliography}{99}
    \bibitem{Tsirulev2020}
    V. V. Nikonov, A. N. Tsirulev. \textit{Pauli basis formalism in quantum computations}. Volume 8, No 3, pp. 1 – 14, 2020.\\
    (\href{https:doi.org/10.26456/mmg/2020-831} {\textit{doi:10.26456/mmg/2020-831}})

    \bibitem{Preskill2018}
    J. Preskill. \textit{Quantum Computing in the NISQ era and beyond}. Quantum, vol. 2, p. 79, 2018.\\
    (\href{https://quantum-journal.org/papers/q-2018-08-06-79/}{\textit{quantum-journal:q-2018-08-06-79}})

    \bibitem{Cerezo2021}
    M. Cerezo, et al. \textit{Variational Quantum Algorithms}. Nature Reviews Physics, vol. 3, pp. 625-644, 2021.\\
    (\href{https://www.nature.com/articles/s42254-021-00348-9}{\textit{nature:42254-021-00348-9}})

    \bibitem{Peruzzo2014}
    A. Peruzzo, et al. \textit{A variational eigenvalue solver on a photonic quantum processor}. Nature Communications, vol. 5, p. 4213, 2014.\\
    (\href{https://www.nature.com/articles/ncomms5213}{\textit{nature:ncomms5213}})

    \bibitem{Farhi2014}
    E. Farhi, J. Goldstone, and S. Gutmann. \textit{A Quantum Approximate Optimization Algorithm}. arXiv preprint arXiv:1411.4028, 2014.\\
    (\href{https://arxiv.org/abs/1411.4028}{\textit{arXiv:1411.4028}})

    \bibitem{McClean2016}
    J. R. McClean, et al. \textit{The theory of variational hybrid quantum-classical algorithms}. New Journal of Physics, vol. 18, p. 023023, 2016.\\
    (\href{https://iopscience.iop.org/article/10.1088/1367-2630/18/2/023023}{\textit{iopscience:1367-2630-18-2-023023}})

    \bibitem{Kandala2017}
    A. Kandala, et al. \textit{Hardware-efficient variational quantum eigensolver for small molecules and quantum magnets}. Nature, vol. 549, pp. 242-246, 2017.\\
    (\href{https://www.nature.com/articles/nature23879}{\textit{nature:nature23879}})

    \bibitem{Harrow2009}
    A. W. Harrow, A. Hassidim, and S. Lloyd. \textit{Quantum algorithm for linear systems of equations}. Physical Review Letters, vol. 103, no. 15, p. 150502, 2009.\\
    (\href{https://journals.aps.org/prl/abstract/10.1103/PhysRevLett.103.150502}{\textit{aps:PhysRevLett.103.150502}})

    \bibitem{Biamonte2017}
    J. Biamonte, et al. \textit{Quantum machine learning}. Nature, vol. 549, pp. 195-202, 2017.\\
    (\href{https://www.nature.com/articles/nature23474}{\textit{nature:nature23474}})

    \bibitem{Lopatin}
    A. A. Lopatin. \textit{Квантовая механика и её приложения}. Санкт-Петербургский Государственный Университет.\\
    (\href{https://math.spbu.ru/user/gran/sb1/lopatin.pdf}{\textit{math.spbu:user/gran/sb1/lopatin}})

    \bibitem{Aspuru-Guzik2005}
    A. Aspuru-Guzik, A. D. Dutoi, P. J. Love, M. Head-Gordon. \textit{Simulated Quantum Computation of Molecular Energies}. Science, vol. 309, no. 5741, pp. 1704-1707, 2005.\\
    (\href{https://www.science.org/doi/10.1126/science.1113479}{\textit{science:1113479}})

    \bibitem{Schuld2015}
    M. Schuld, I. Sinayskiy, F. Petruccione. \textit{An introduction to quantum machine learning}. Contemporary Physics, vol. 56, no. 2, pp. 172-185, 2015.\\
    (\href{https://www.tandfonline.com/doi/abs/10.1080/00107514.2014.964942}{\textit{tandfonline:00107514.2014.964942}})

    \bibitem{Daskin2014}
    A. Daskin, S. Kais. \textit{Decomposition of unitary matrices for finding quantum circuits: Application to molecular Hamiltonians}. The Journal of Chemical Physics, vol. 141, no. 23, p. 234115, 2014.\\
    (\href{https://aip.scitation.org/doi/10.1063/1.4904315}{\textit{aip:1.4904315}})

    \bibitem{Romero2018}
    J. Romero, R. Babbush, J. R. McClean, C. Hempel, P. J. Love, A. Aspuru-Guzik. \textit{Strategies for quantum computing molecular energies using the unitary coupled cluster ansatz}. Quantum Science and Technology, vol. 4, no. 1, p. 014008, 2018.\\
    (\href{https://iopscience.iop.org/article/10.1088/2058-9565/aad3e4}{\textit{iopscience:2058-9565/aad3e4}})

    \bibitem{Havlicek2019}
    V. Havlicek, A. D. Córcoles, K. Temme, A. W. Harrow, A. Kandala, J. M. Chow, J. M. Gambetta. \textit{Supervised learning with quantum-enhanced feature spaces}. Nature, vol. 567, pp. 209-212, 2019.\\
    (\href{https://www.nature.com/articles/s41586-019-0980-2}{\textit{nature:s41586-019-0980-2}})

    \bibitem{Moll2018}
    N. Moll, P. Barkoutsos, L. Bishop, J. M. Chow, A. Cross, D. J. Egger, S. Filipp, A. Fuhrer, J. M. Gambetta, M. Ganzhorn, et al. \textit{Quantum optimization using variational algorithms on near-term quantum devices}. Quantum Science and Technology, vol. 3, no. 3, p. 030503, 2018.\\
    (\href{https://iopscience.iop.org/article/10.1088/2058-9565/aab822}{\textit{iopscience:2058-9565/aab822}})
\end{thebibliography}

%#########################################################################################################################################################################################################################
%################################################################################################### Приложение ##########################################################################################################
%#########################################################################################################################################################################################################################

\addcontentsline{toc}{chapter}{\hspace{5.5mm} Приложение  C{$\#$}}
\chapter*{Приложение C{$\#$}}

$${\hat{U}(\theta) = e^{i\theta_1\sigma_{k_1}}}e^{i\theta_2\sigma_{k_2}}\ldots e^{i\theta_{m} \sigma_{k_m}}$$
$$(cos(\theta_m)\hat{\sigma}_0 + isin(\theta_m)\hat{\sigma_{k_m}})$$

\end{document} 