\documentclass[a4paper]{report}

\def\baselinestretch{1.1}
\usepackage[14pt]{extsizes}
\usepackage[utf8]{inputenc}
\usepackage[russian]{babel}
\usepackage{indentfirst}
\usepackage{mathrsfs}

%%%%%%%%%%%%%%%%% Символы, графика %%%%%%%%%%%%%%%%%%%%%

\usepackage[T2A]{fontenc}
\usepackage{amsmath,amssymb,amsfonts,amsthm}
\usepackage{bm}
\newcommand\dsone{\mathds{H}}
\usepackage{graphicx}
\usepackage{color}
\usepackage[pdftex,colorlinks,linkcolor=blue,citecolor=blue]{hyperref}
\usepackage{pgfplots}
\usepackage{tikz}
\usepackage{array}
\newcolumntype{P}[1]{>{\centering\arraybackslash}p{#1}}

%%%%%%%%% Разметка страницы %%%%%%%%%

\bibliographystyle{plain}  % Change this to your preferred style
\renewcommand{\thetable}{\arabic{table}}
\usepackage{indentfirst}
\topmargin=-1.5cm %отступ сверху
\oddsidemargin=0.4cm %отступ слева (нечетные страницы)
\evensidemargin=0.4cm %(четные страницы)
\textwidth=16cm %ширина текста
\textheight=24cm
\tolerance=800
\parskip=1ex

\pagestyle{plain}

\usepackage{listings}

\definecolor{codegreen}{rgb}{0,0.6,0}
\definecolor{codegray}{rgb}{0.5,0.5,0.5}
\definecolor{codepurple}{rgb}{0.58,0,0.82}
\definecolor{backcolour}{rgb}{0.95,0.95,0.92}

\lstset{
    extendedchars=true,  % Corrected this line
    backgroundcolor=\color{backcolour},
    commentstyle=\color{codegreen},
    keywordstyle=\color{magenta},
    numberstyle=\tiny\color{codegray},
    stringstyle=\color{codepurple},
    breakatwhitespace=false,
    breaklines=true,
    captionpos=b,
    keepspaces=true,
    numbers=left,
    numbersep=3pt,
    showspaces=false,
    showstringspaces=false,
    showtabs=false,
    tabsize=1,
    basicstyle=\fontsize{10}{12}\selectfont\ttfamily
}

%%%%%%%% newcommands:  ket,  bra,  ketbra,  braket
\newcommand{\ket}[1] {\!\!\;\ensuremath{\left|#1\right\rangle}}
\newcommand{\bra}[1] {\!\!\:\ensuremath{\left\langle#1\right|\!\!\:}}
\newcommand{\ketbra}[2]{\!\!\:\ensuremath {\left|#1\right\rangle\!\:\!\!\left\langle#2\right|}}
\newcommand{\braket}[2]{\ensuremath {\!\!\:\left\langle#1\!\!\: \left|\!\!\!\;\right.#2\right\rangle\!\!\;}}



\begin{document}

\begin{titlepage}
	\begin{center}
		Министерство науки и высшего образования РФ\\
		ФГБОУ ВО «Тверской государственный университет»\\
		Математический факультет\\
		Направление 02.04.01 Математика и компьютерные науки\\
		Профиль <<Математическое и компьютерное моделирование>>	
	\end{center}
	
	\vspace{1.4cm}
	\begin{center}
	
		{МАГИСТЕРСКАЯ ДИССЕРТАЦИЯ}	
		
		\vspace{1.0cm}
    \large{Вариационный квантовый алгоритм с оптимизацией методом отжига}
		
		
		\vspace{1.0cm}
	\end{center}
	
	
	
	\begin{flushright}
		\begin{minipage}{80mm}
			Автор:\\
			Алешин Д.А.\\
      Подпись:
			
			\vspace{1.0cm}
			Научный руководитель:\\
			д. ф.-м. н. Цирулёв А.Н.\\
      Подпись:
			
		\end{minipage}
	\end{flushright}
	
	
	\vspace{1.6cm}
	\noindent Допущен к защите:\\
	Руководитель ООП: Цветков В.П.\\[0.3cm]
  $\underset{\textit{(подпись, дата)}}{\underline{\hspace{0.3\textwidth}}}$
	\vspace{2.2cm}
	
	
	
	\begin{center}
		Тверь 2025
	\end{center}
	
	\date{}
\end{titlepage}

\setcounter{page}{2}

\tableofcontents
\newpage

% Abstract
\addcontentsline{toc}{chapter}{\hspace{5.5mm} Введение}
\chapter*{Введение}

\chapter{Общая схема квантовых вариационных алгоритмов}

\section{Базис Паули}
Матрицы Паули, обозначаемые как $\sigma_x$, $\sigma_y$ и $\sigma_z$, представляют собой набор эрмитовых и унитарных $2 \times 2$ матриц. Эти матрицы играют центральную роль в описании квантовых систем с полуцелым спином, таких как электроны, и занимают важное место в теории представлений группы SU(2). Их использование охватывает широкий спектр задач в квантовой механике и квантовой теории поля.

Матрицы Паули определяются следующим образом:

$$
\sigma_x = \begin{pmatrix}
0 & 1 \\
1 & 0
\end{pmatrix}, \quad
\sigma_y = \begin{pmatrix}
0 & -i \\
i & 0
\end{pmatrix}, \quad
\sigma_z = \begin{pmatrix}
1 & 0 \\
0 & -1
\end{pmatrix}.
$$

Эти матрицы имеют ряд уникальных свойств, включая эрмитовость ($\sigma_i = \sigma_i^\dagger$) и унитарность ($\sigma_i \sigma_i^\dagger = I$). Они также удовлетворяют специфическим соотношениям коммутации и антикоммутации, что делает их полезными для описания квантовых преобразований и взаимодействий.

Коммутационные соотношения для матриц Паули выражаются следующим образом:

$$
[\sigma_i, \sigma_j] = 2i\epsilon_{ijk}\sigma_k,
$$

где $\epsilon_{ijk}$ — символ Леви-Чивиты. Эти соотношения играют важную роль в понимании квантовой динамики и описании вращений в спиновых системах.

Антикоммутационные соотношения для матриц Паули выглядят следующим образом:

$$
\{\sigma_i, \sigma_j\} = 2\delta_{ij}I,
$$

где $\delta_{ij}$ — символ Кронекера, а $I$ — единичная матрица. Эти свойства широко используются в квантовых вычислениях и других приложениях, где важна как коммутация, так и антикоммутация операторов.

Матрицы Паули находят применение в описании операторов спина. Оператор спина частицы можно выразить через линейную комбинацию матриц Паули:

$$
\vec{S} = \frac{\hbar}{2} \vec{\sigma},
$$

где $\vec{S}$ — оператор спина, $\hbar$ — приведенная постоянная Планка. Это позволяет моделировать взаимодействие спина с внешними полями и другими квантовыми системами.

В квантовой информации матрицы Паули формируют базис для всех эрмитовых матриц размерности $2 \times 2$. Любой эрмитов оператор может быть представлен как:

$$
A = a_0 I + a_x \sigma_x + a_y \sigma_y + a_z \sigma_z,
$$

где $a_0, a_x, a_y, a_z$ — вещественные коэффициенты. Это представление используется в квантовой томографии и других вычислительных задачах.




\section{Вариационная квантовая оптимизация}

Пусть $\mathcal{H}_n$ --- гильбертово пространство квантовой системы, состоящей из $n$ кубитов, ${\mathcal{S}\subset\mathcal{H}_n}$ --- пространство векторов состояния (т.е. векторов, нормированных на единицу), $L(\mathcal{H}_n)$ --- алгебра операторов на $\mathcal{H}_n$ и ${\hat{H}\in L(\mathcal{H}_n)}$ --- эрмитов оператор. Определим функцию
\begin{equation}\label{E(u)}
  E(\ket{u})=\bra{u}\hat{H}\ket{u},\quad \ket{u}\in\mathcal{S}.
\end{equation}
\textit{В простейшей постановке \textbf{квантовой задачи оптимизации} требуется найти вектор состояния, на котором целевая функция (функция стоимости) $E$ принимает минимальное значение}, т.е., в формальной записи, решить задачу
\begin{equation}\label{Emin}
  E(\ket{u})\; \stackrel{\ket{u}\in\mathcal{S}}{-\!-\!\!\!\longrightarrow}\; \min.
\end{equation}
Ниже, для определенности и краткости, будем называть $\hat{H}$ гамильтонианом системы, а целевую функцию $E$ --- энергией.

Сложность алгоритмов прямого вычисления собственных значений гамильтониана $\hat{H}$ растет экспоненциально с ростом числа кубитов, поэтому для больших систем используются вариационные методы решения задачи оптимизации~(\ref{Emin}).

Вариационными квантовыми алгоритмами обычно называют такие гибридные квантово-классические алгоритмы, нацеленные на решение квантовых задач оптимизации посредством квантовых вычислений или их классической имитации, которые проводят вариационную настройку параметров квантовой схемы. Параметрически управляемое квантовое устройство, обычно представленное квантовой цепью, реализует анзац, т.е. унитарное преобразование стандартного начального состояния ${\ket{0}^{\!\otimes\:\! n}}$ или, как вариант, предудущего полученного состояния. На каждом шаге регулирующие параметры подбираются так, чтобы минимизировать энергию (целевую функцию). Обычно это выполняется путём измерения энергии состояний, предоставляемых вариационной схемой, и обновления параметров для минимизации целевой функции.

В точной математической формулировке сказанное означает, что в функции~(\ref{E(u)}) вектор состояния $\ket{u}$ зависит от набора $m$ параметров $\bm\theta=(\theta_1, \ldots,\theta_m)$, которые принимают значения в некоторой связной и односвязной области $\Omega\in\mathbb{R}^m$. Вариационная формулировка квантовой задачи оптимизации~(\ref{Emin}) имеет вид
\begin{equation}\label{Emin-theta}
  E\big(\ket{u(\bm\theta)}\big)\; \stackrel{\bm\theta\,\in\,\Omega}{-\!-\!\!\!\longrightarrow}\; \min, \qquad E\big(\ket{u(\bm\theta)}\big)=\bra{u(\bm\theta)}\hat{H}\ket{u(\bm\theta)}.
\end{equation}

Итак, цель вариационного квантового алгоритма --- найти такой набор параметров, на котором энергия достигает минимума. Число параметров $m$ в наборе $\bm\theta=(\theta_1, \ldots,\theta_m)$ зависит от конкретной задачи, в частности, от числа кубитов в квантовом устройстве. Для $n$ кубитов размерность пространства состояний, $N=2^n$, растет экспоненциально с ростом числа кубитов. Поэтому вариационный квантовый алгоритм должен быть организован и выполнен так, чтобы выполнялось условие $m\ll N$, поскольку в противном случае высокий класс сложности алгоритма сделает его неэффективным с практической точки зрения.

Но наиболее важным вопросом является выбор зависимости вектора состояния $\ket{u(\bm\theta)}$ от параметров. В вариационных квантовых алгоритмах используется \textit{анзац} (унитарное преобразование) вида
\begin{equation}\label{u(theta)}
\ket{u(\bm\theta)}= \hat{U}(\bm\theta)\ket{u_0}.
\end{equation}
В общем случае форма анзаца определяет, какими будут параметры $\boldsymbol\theta$ и как их можно настроить для минимизации энергии (целевой функции). Структура анзаца, как правило, будет зависеть от поставленной задачи, так как во многих случаях можно использовать информацию о проблеме, чтобы подобрать анзац: это "анзац, подсказнный задачей". Однако можно построить анзацы достаточно общего вида, которые пригодны для использования в некоторых классах задач даже тогда, когда интуиция и известная информация о задаче не позволят его уточнить. Стандартно анзац выбирается в виде композиции $m$ последовательно примененных унитарных преобразований
\begin{equation}\label{Ansatz}
\hat{U}(\boldsymbol\theta)= \hat{U}_{m}(\theta_m)\cdots \hat{U}_{1}(\theta_1).
\end{equation}

В композиции~(\ref{Ansatz}) выбор операторов определяется типом задачи и технической возможностью их реализации на конкретном квантовом устройстве. Например, можно выбрать
\begin{equation}\label{Ansatz1}
\hat{U}_{K}(\theta_K)= \hat{W}_K\exp\!\big(i\theta_K\hat{\sigma}_K\big)= \hat{W}_K\big(\cos\:\!\!\theta_K\hat{\sigma}_{0\ldots0}+ i\sin\:\!\!\theta_K\hat{\sigma}_K\big),
\end{equation}
где $1\!\leqslant\! K\!\leqslant\! m,\; \hat{\sigma}_K\!=\! \hat{\sigma}_{k_1}\!\otimes\ldots\otimes\hat{\sigma}_{k_n},\; k_1,\!\ldots,\:\!\!k_n\!\in\!\{0,1,2,3\},\; K\!=\!k_1\!\ldots k_n$\linebreak
(т.е. $K$ --- десятичное представление строки $k_1\ldots k_n$, рассматриваемой как число по основанию 4), $n$ --- число кубитов, а $\hat{W}_K$ --- независящий от параметров унитарный оператор. Как правило, в строке $k_1\ldots k_n$ только отдельные числа отличны от нуля, так что в тензорном произведении ${\hat{\sigma}_{k_1}\!\otimes\ldots \otimes \hat{\sigma}_{k_n}}$ часть операторов являются тождественными.

В другом распространенном варианте операторы в композиции~(\ref{Ansatz}) имеют вид
\begin{equation}\label{Ansatz2}
\hat{U}_{K}(\theta_K)= \hat{W}_K
\big(\mathrm{e}^{i\theta_{k_1}\hat{\sigma}_{k_1}} \otimes\ldots\otimes \mathrm{e}^{i\theta_{k_n}\hat{\sigma}_{k_n}}\big),
\end{equation}
где по-прежнему $1\!\leqslant\! K\!\leqslant\! m$ и $K\!=\!k_1\!\ldots k_n$. Если в~(\ref{Ansatz1}) все операторы $\hat{W}_K$ могут быть тождественными, то в~(\ref{Ansatz2}), по крайней мере некоторые операторы $\hat{W}_K$ должны быть запутывающими и, следовательно, как минимум двухкубитными.

Таким образом, анзацы~(\ref{Ansatz}), (\ref{Ansatz1}) и (\ref{Ansatz2}) конкретизируют вариационную квантовую задачу оптимизации~(\ref{Emin-theta}) и~(\ref{u(theta)}) в отношении параметрической зависимости вектора состояния,
\begin{equation*}
\ket{u(\bm\theta)}= \hat{U}(\bm\theta)\ket{0\ldots0},
\end{equation*}
где начальное состояние имеет вид ${\ket{0\ldots0}=\ket{0}^{\!\otimes\:\! n}}$. Если предположить далее, что мы в состоянии уверенно приготовить начальное состояние, реализовать анзац на физическом устройстве и вычислить значение энергии $E\big(\ket{u(\bm\theta)}\big)=\bra{u(\bm\theta)}\hat{H}\ket{u(\bm\theta)}$ посредством измерений (с привлечением классического компьютера), то следующий --- основной --- вопрос можно сформулировать так: как искать параметры, которые обеспечивают глобальный минимум энергии. Этот этап выполняется с помощью классического компьютера, так что вариационный квантовый алгоритм --- гибридный квантово-классический алгоритм: параметризованная квантовая схема и измерительный прибор представляют квантовую часть, а алгоритм настройки параметров — классическую.

\section{Общая схема алгоритма и анзац}



\section{Пример, иллюстрирующий особенности алгоритма}


Для иллюстрации алгоритма рассмотрим гамильтониан
\begin{align}\label{H}
\hat{H} = 2\,\hat{\sigma}_{03} + \hat{\sigma}_{30} - 4\,\hat{\sigma}_{11},
\end{align}
который в стандартном базисе $\big\{\ket{00}, \ket{01}, \ket{10}, \ket{11}\!\big\}$ имеет матрицу
\begin{equation}\label{}
H=
\!\!\left(\!
\begin{array}{cccc}
3& 0& 0& \!\!-4 \vphantom{\hat{A}}  \\[2pt]
0& \!\!-1& \!\!-4& 0 \vphantom{\hat{A}}  \\[2pt]
0& \!\!-4& 1& 0 \vphantom{\hat{A}}  \\[2pt]
\!\!-4& 0& 0& \!\!-3 \vphantom{\hat{A}}
\end{array}
\!\!\right).\nonumber
\end{equation}
Используя систему Maple находим собственные значения и собственные состояния в порядке возрастания собственных значений, начиная с основного состояния $\ket{u_0}$ с собственным значением $E_0$:
\begin{align}
E_0&= -5,\; &&\ket{u_0}= \frac{1}{\,\sqrt{5}\,}\,\ket{00}+ \frac{2}{\,\sqrt{5}\,}\,\ket{11}, \label{E0-u0}\\
E_1&= -\sqrt{17},\;\vphantom{\int\limits^A} &&\ket{u_1}= \frac{\sqrt{17}+1}{\,\sqrt{34+2\sqrt{17}\,}\,}\,\ket{01}+ \frac{\sqrt{8}}{\,\sqrt{17+\sqrt{17}\,}\,}\,\ket{10}, \label{E1-u1}\\
E_2&= \sqrt{17},\;\vphantom{\int\limits^A} &&\ket{u_1}= -\frac{\sqrt{17}-1}{\,\sqrt{34-2\sqrt{17}\,}\,}\,\ket{01}+ \frac{\sqrt{8}}{\,\sqrt{17-\sqrt{17}\,}\,}\,\ket{10}, \label{E2-u2}\\
E_3&= 5,\;\vphantom{\int\limits^A} &&\ket{u_3}= -\frac{2}{\,\sqrt{5}\,}\,\ket{00}+ \frac{1}{\,\sqrt{5}\,}\,\ket{11}. \label{E3-u3}
\end{align}
Рассмотрим далее пошаговое выполнение вариационного квантового алгоритма, который позволяет найти состояние, близкое к основному.

\textit{Первый шаг --- выбор анзаца}, т.е. унитарного преобразования $\hat{U}(\bm\theta)$. В гамильтониан~(\ref{H}) не входят операторы вида $\hat{\sigma}_{k2}$ и $\hat{\sigma}_{k2}$ с $k\neq2$, поэтому имеет смысл сразу выбирать анзац так, чтобы при действии на $k\neq2$ он давал вектор состояния с вещественными коэффициентами. Других наводящих соображений относительно формы анзаца не видно, поэтому следует рассмотреть разные варианты. В общем случае вектор параметров $\bm\theta$ четырехмерен. В простейшем варианте анзац с четырехмерным вектором параметров $\bm\theta=(\xi,\lambda,\mu,\nu)$ можно выбирать как композицию экспонент
\begin{equation}\label{ansatz}
\hat{U}(\bm\theta)= \mathrm{e}^{i\xi\hat{\sigma}_{02}}\mathrm{e}^{i\lambda\hat{\sigma}_{03}} \mathrm{e}^{i\mu\hat{\sigma}_{30}} \mathrm{e}^{i\nu\hat{\sigma}_{11}}
\end{equation}
операторов Паули, присутствующих в гамильтониане~(\ref{H}). Вычислим вначале
\begin{multline}\label{}
\mathrm{e}^{i\mu\hat{\sigma}_{30}} \mathrm{e}^{i\nu\hat{\sigma}_{11}}\ket{00}= \big(\cos\:\!\!\mu\,\hat{\sigma}_{00}+ i\sin\:\!\!\mu\,\hat{\sigma}_{30}\big)\big(\cos\:\!\!\nu\ket{00}+ i\sin\:\!\!\nu\ket{11}\big)
\vphantom{\int}\\
=\cos\:\!\!\mu\cos\:\!\!\nu\ket{00}- \sin\:\!\!\mu\sin\:\!\!\nu\ket{11}+ i\sin\:\!\!\mu\cos\:\!\!\nu\ket{00}+ i\cos\:\!\!\mu\sin\:\!\!\nu\ket{11}
\vphantom{\int}\\
=\mathrm{e}^{i\mu}\cos\:\!\!\nu\ket{00}+ i\mathrm{e}^{i\mu}\sin\:\!\!\nu\ket{11}= \mathrm{e}^{i\mu}\cos\:\!\!\nu\ket{00}+ \mathrm{e}^{i(\mu+\pi/2)}\sin\:\!\!\nu\ket{11}.
\nonumber
\end{multline}
Действуя на результат оператором $\mathrm{e}^{i\lambda\hat{\sigma}_{03}}$, получим следующий промежуточный вектор состояния:
\begin{multline}\label{3param}
\mathrm{e}^{i\lambda\hat{\sigma}_{03}} \mathrm{e}^{i\mu\hat{\sigma}_{30}} \mathrm{e}^{i\nu\hat{\sigma}_{11}}\ket{00}= \mathrm{e}^{i\mu}\cos\:\!\!\nu \big(\cos\:\!\!\lambda\,\hat{\sigma}_{00}+ i\sin\:\!\!\lambda\hat{\sigma}_{03}\big)
\ket{00}\\
\qquad\qquad\qquad\qquad\qquad\quad\: +\mathrm{e}^{i(\mu+\pi/2)}\sin\:\!\!\nu \big(\cos\:\!\!\lambda\,\hat{\sigma}_{00}+ i\sin\:\!\!\lambda\hat{\sigma}_{03}\big)
\ket{11} \vphantom{\int}\\
\;\;=\mathrm{e}^{i\mu}\cos\:\!\!\nu
\big(\cos\:\!\!\lambda+ i\sin\:\!\!\lambda\big)\ket{00}+
\mathrm{e}^{i(\mu+\pi/2)}\sin\:\!\!\nu
\big(\cos\:\!\!\lambda+ i\sin\:\!\!\lambda\big)\ket{11}
\vphantom{\int}\\
=\mathrm{e}^{i(\mu+\lambda)}\cos\:\!\!\nu\ket{00}+ \mathrm{e}^{i(\mu+\pi/2-\lambda)}\sin\:\!\!\nu\ket{11}.\quad\,
\end{multline}
Очевидно, что этот анзац не является универсальным.

Варьируя параметры $\lambda,\mu,\nu$, можно получить основное состояние~(\ref{E0-u0}) с точностью до несущественного множителя $\mathrm{e}^{i(\mu+\pi/4)}$, например, при
\begin{equation}\label{param}
\lambda=\pi/4, \quad \mu\in\mathbb{R}, \quad \cos\:\!\!\nu=1/\sqrt{5},
\quad
\sin\:\!\!\nu=2/\sqrt{5}.
\end{equation}
Здесь $\mu$ --- любое, поэтому к нужному результату приводит более простой анзац (при $\mu=0$) $\hat{U}(\bm\theta)= \mathrm{e}^{i\lambda\hat{\sigma}_{03}} \mathrm{e}^{i\nu\hat{\sigma}_{11}}$, однако заранее это нам не известно. Более того основное состояние~(\ref{E0-u0}) можно достигнуть (что заранее также неизвестно и неочевидно) даже однопараметрическим анзацем
\begin{equation*}\label{}
\mathrm{e}^{i\nu\hat{\sigma}_{12}}\ket{00}= \big(\cos\:\!\!\nu\,\hat{\sigma}_{00}+ i\sin\:\!\!\nu\,\hat{\sigma}_{12}\big)\ket{00}= \cos\:\!\!\nu\ket{00}+ \sin\:\!\!\nu\ket{11},
\nonumber
\end{equation*}
с теми же значениями $\cos\:\!\!\nu$ и $\sin\:\!\!\nu$, что и в~(\ref{param}).

Действуя на~(\ref{3param}) оператором $\mathrm{e}^{i\xi\hat{\sigma}_{02}}$, получим вектор состояния
\begin{multline}\label{Phi0}
\ket{\Phi}= \hat{U}(\bm\theta)\ket{00}\;\\
=\big(\cos\:\!\!\xi\,\hat{\sigma}_{00}+ i\sin\:\!\!\xi\,\hat{\sigma}_{02}\big) \big(\mathrm{e}^{i(\mu+\lambda)}\cos\:\!\!\nu\ket{00}+ \mathrm{e}^{i(\mu+\pi/2-\lambda)}\sin\:\!\!\nu\ket{11}\big)\! \vphantom{\int}\\
=\mathrm{e}^{i(\mu+\lambda)}\sin\:\!\!\xi\cos\:\!\!\nu\ket{01}- \mathrm{e}^{i(\mu+\pi/2-\lambda)}\sin\:\!\!\xi\sin\:\!\!\nu\ket{10}
\hspace{6.5em} \vphantom{\int}\\
+\mathrm{e}^{i(\mu+\lambda)}\cos\:\!\!\xi\cos\:\!\!\nu\ket{00}+ \mathrm{e}^{i(\mu+\pi/2-\lambda)}\cos\:\!\!\xi\sin\:\!\!\nu\ket{11},\;\,
\end{multline}
который зависит от четырех параметров. Из формы данного вектора видно, что анзац~(\ref{ansatz}) универсален (с учетом замечания о вещественности коэффициентов, сделанного выше). Основное состояние достигается при произвольном $\mu\in\mathbb{R}$ и
\begin{equation}\label{param1}
\!\!\xi=0, \;\lambda=\!\big\{\pi/4,7\pi/4\big\}, \; \cos\:\!\!\nu=1/\sqrt{5},
\; \sin\:\!\!\nu=\big\{2/\sqrt{5},-2/\sqrt{5}\big\}\\
\end{equation}
или
\begin{equation}\label{param2}
\,\xi=\pi, \;\lambda=\!\big\{\pi/4,7\pi/4\big\}, \; {\cos\:\!\!\nu}=\!-1/\sqrt{5},
\; {\sin\:\!\!\nu}=\big\{\!\!-\:\!\!2/\sqrt{5},2/\sqrt{5}\big\}.
\end{equation}

Для сокращения записи имеет смысл освободиться в~(\ref{Phi0}) от фазового множителя и записать вектор состояния в виде
\begin{multline}\label{Phi}
\ket\Phi= \sin\:\!\!\xi \big(\cos\:\!\!\nu\ket{01}- \mathrm{e}^{i(\pi/2-2\lambda)}\sin\:\!\!\nu\ket{10}\!\big)
\\
+\cos\:\!\!\xi \big(\cos\:\!\!\nu\ket{00}+ \mathrm{e}^{i(\pi/2-2\lambda)}\sin\:\!\!\nu\ket{11}\!\big)
\end{multline}

\textit{Второй шаг --- вычисление энергии состояния}, т.е. среднего значения $\bra\Phi{\hat{H}}\ket\Phi$. Заметим, что первый и второй шаги должны выполняться на квантовых устройствах, а при классической симуляции алгоритма необходимо проводить явные вычисления. Из~(\ref{H}) и~(\ref{Phi}) находим
\begin{multline}\label{}
{\hat{H}}\ket\Phi= \sin\:\!\!\xi\big(2\,\hat{\sigma}_{03}+ \hat{\sigma}_{30}- 4\hat{\sigma}_{11}\big)\! \big(\cos\:\!\!\nu\ket{01}- \mathrm{e}^{i(\pi/2-2\lambda)}\sin\:\!\!\nu\ket{10}\!\big)
\\
+\cos\:\!\!\xi\big(2\,\hat{\sigma}_{03}+ \hat{\sigma}_{30}- 4\hat{\sigma}_{11}\big)\! \big(\cos\:\!\!\nu\ket{00}+ \mathrm{e}^{i(\pi/2-2\lambda)}\sin\:\!\!\nu\ket{11}\!\big)
\hspace{0.2em} \vphantom{\int\limits^A_A}
\\%%%%%%%%%%%%%% |01> |10>
={\sin\:\!\!\xi}\!\left\{\! \big(4\mathrm{e}^{i(\pi/2-2\lambda)}\sin\:\!\!\nu- \cos\:\!\!\nu\big)\ket{01}\right.\hspace{7.6em}
\\
\hspace{16em}\left.-\big(\mathrm{e}^{i(\pi/2-2\lambda)}\sin\:\!\!\nu+ 4\cos\:\!\!\nu\big)\ket{10}\!\right\}
\\%%%%%%%%%%%%%% |00> |11>
+{\cos\:\!\!\xi}\!\left\{\! \big(3\cos\:\!\!\nu-4\mathrm{e}^{i(\pi/2-2\lambda)}\sin\:\!\!\nu\big)\ket{00} \right.\hspace{6.6em}
\\
\left.-\big(4\cos\:\!\!\nu+ 3\mathrm{e}^{i(\pi/2-2\lambda)}\sin\:\!\!\nu\big) \ket{11}\!\right\}\!.\hspace{0.5em}
\nonumber
\end{multline}
Поскольку
\begin{multline}\label{}
\bra\Phi=
\sin\:\!\!\xi \big(\cos\:\!\!\nu\bra{01}- \mathrm{e}^{-i(\pi/2-2\lambda)}\sin\:\!\!\nu\bra{10}\,\big)
\\
+\cos\:\!\!\xi \big(\cos\:\!\!\nu\bra{00}+ \mathrm{e}^{-i(\pi/2-2\lambda)}\sin\:\!\!\nu\bra{11}\,\big),
\hspace{2.2em} \vphantom{\int\limits^A}
\nonumber
\end{multline}
то
\begin{multline}\label{EPhi}
E_{\Phi}= \bra\Phi{\hat{H}}\ket\Phi
\\
=\sin^2\:\!\!\xi\!\left\{\!\cos\:\!\!\nu \big(4\mathrm{e}^{i(\pi/2-2\lambda)}\sin\:\!\!\nu- \cos\:\!\!\nu\big) \right.\hspace{7.6em}
\\
\hspace{16em}\left.+\sin\:\!\!\nu\big(\sin\:\!\!\nu+ 4\mathrm{e}^{-i(\pi/2-2\lambda)}\cos\:\!\!\nu\big)\right\}
\\%%%%%%%%%%%%%% |00> |11>
+\cos^2\:\!\!\xi\!\left\{\!\cos\:\!\!\nu\big(3\cos\:\!\!\nu- 4\mathrm{e}^{i(\pi/2-2\lambda)}\sin\:\!\!\nu\big) \right.\hspace{6.6em}
\\
\hspace{4em}\left.-\sin\:\!\!\nu\big(4\mathrm{e}^{-i(\pi/2-2\lambda)}\cos\:\!\!\nu+ 3\sin\:\!\!\nu\big)\!\right\}
\vphantom{\underbrace{j_|}}
\\
=\sin^2\:\!\!\xi\big(4\sin\:\!\!2\lambda\sin\:\!\!2\nu- \cos\:\!\!2\nu\big)+
\cos^2\:\!\!\xi\big(3\cos\:\!\!2\nu- 4\sin\:\!\!2\lambda\sin\:\!\!2\nu\big).
\vphantom{\widehat{A^|}}
\end{multline}
Разумеется, если взять значения ${\xi,\, \lambda,\, \cos\:\!\!\nu,\, \sin\:\!\!\nu}$ как в~(\ref{param1}) или в~(\ref{param2}), то мы получим энергию основного состояния~(\ref{E0-u0}), т.е.  $E_{\Phi}=-5$.


\textit{Третий шаг --- изменение значений параметров} $\lambda,\mu,\nu$ (с целью минимизации $E_{\Phi}$) и возвращение к первому шагу; предполагается, что в начале выполнения алгоритма начальные значения параметров заданы. Из~(\ref{EPhi}) видно, что на значение $E_{\Phi}$ параметр $\mu$ не влияет, а параметры $\lambda,\nu$ должны варьироваться в области ${[0,\pi]\times[0,\pi]}$. Однако изначально это неизвестно, поэтому все четыре параметра должны варьироваться в области ${[0,2\pi]\times[0,2\pi]\times[0,2\pi]\times[0,2\pi]}$. Имеет смысл установить независимость энергии от параметра $\mu$ (и ее зависимость от остальных параметров) в начале работы алгоритма.

Мы уже знаем, что глобальный минимум энергии достигается для четырех наборов параметров~(\ref{param1}) и~(\ref{param2}). Соответствующие собственные векторы, вычисленные по выражению~(\ref{Phi}) отличаются от~(\ref{E0-u0}) только фазовыми множителями. Теперь необходимо выяснить, имеются ли у функции (трех переменных)~(\ref{EPhi}) другие локальные минимумы.

Используя систему Maple, вычислим производные
\begin{equation*}
\partial_\xi E_{\Phi},\,\; \partial_\lambda E_{\Phi},\,\; \partial_\nu E_{\Phi},\,\;
\end{equation*}
\begin{equation*}
A=\partial_\xi^{\:\!2}E_{\Phi},\,\; B=\partial_\lambda^{\:\!2}E_{\Phi},\,\; C=\partial_\nu^{\:\!2}E_{\Phi},\,\;
\end{equation*}
\begin{equation*}
K= \partial_{\xi\lambda}E_{\Phi},\,\; L=\partial_{\xi\nu}E_{\Phi},\,\; M=\partial_{\lambda\nu}E_{\Phi}.\,\;
\end{equation*}
Находим
\begin{eqnarray*}
\partial_\xi E_{\Phi} &=& 4\sin\:\!\!2\xi\big(2\sin\:\!\!2\lambda\sin\:\!\!2\nu- \cos\:\!\!2\nu\big),
\\
\partial_\lambda E_{\Phi} &=& -8\cos\:\!\!2\xi\cos\:\!\!2\lambda\sin\:\!\!2\nu, \\
\partial_\nu E_{\Phi} &=& \sin^2\:\!\!\xi\big(8\sin\:\!\!2\lambda\cos\:\!\!2\nu+ 2\sin\:\!\!2\nu\big)-
\cos^2\:\!\!\xi\big(6\sin\:\!\!2\nu+ 8\sin\:\!\!2\lambda\cos\:\!\!2\nu\big),
\\
A &=& 8\cos\:\!\!2\xi \big(2\sin\:\!\!2\lambda\sin\:\!\!2\nu-\cos\:\!\!2\nu\big),
\\
B &=& 16\cos\:\!\!2\xi \sin\:\!\!2\lambda \sin\:\!\!2\nu,
\\
C &=& 4\cos^2\:\!\!\xi\big(4\sin\:\!\!2\lambda\sin\:\!\!2\nu- 3\cos\:\!\!2\nu\big) -4\sin^2\:\!\!\xi\big(4\sin\:\!\!2\lambda\sin\:\!\!2\nu- \cos\:\!\!2\nu\big),
\\
K &=& 16\sin\:\!\!2\xi \cos\:\!\!2\lambda \sin\:\!\!2\nu,
\\
L &=& 4\sin\:\!\!2\xi \big(4\sin\:\!\!2\lambda\cos\:\!\!2\nu+ 2\sin\:\!\!2\nu\big),
\\
M &=& -16\cos\:\!\!2\xi \cos\:\!\!2\lambda \cos\:\!\!2\nu.
\end{eqnarray*}
Необходимые и достаточные условия минимума имеют вид
\begin{equation}\label{ExtrConds}
\partial_\xi E_{\Phi}=0,\quad \partial_\lambda E_{\Phi}=0,\quad \partial_\nu E_{\Phi}=0,
\nonumber
\end{equation}
\begin{equation}\label{}
A>0,\qquad
\det\!\left(\!\!
\begin{array}{cc}
A&\!\! K \\
K&\!\! B\vphantom{\hat{A}}
\end{array}
\!\!\!\:\right)\!>0,
\qquad
\det\!\!\!\:\left(\!\!
\begin{array}{ccc}
A&\! K&\! L \\[2pt]
K&\! B&\! M\vphantom{\hat{A}}  \\[2pt]
L&\! M&\! C\vphantom{\hat{A}}
\end{array}
\!\!\right)\!>0.
\nonumber
\end{equation}

Снова проводя вычисления с помощью системы Maple, обнаруживаем четыре точки локального минимума с энергией ${E_\Phi=-\sqrt{17}}$:
\begin{align*}
\xi=\frac{\pi}{2}, &\qquad\lambda=\frac{\pi}{4}, \qquad \nu=\pi-\,\frac{1}{2}\,\arctan(4),\\
\xi=\frac{\pi}{2}, &\qquad\lambda=\frac{\pi}{4}, \qquad\:\! \!\!\:\nu=2\pi-\,\frac{1}{2}\,\arctan(4),\\
\xi=\frac{\pi}{2}, &\qquad\lambda=\frac{3\pi}{4}, \quad\;\, \nu=\frac{1}{2}\,\arctan(4),\\
\xi=\frac{\pi}{2}, &\qquad\lambda=\frac{3\pi}{4}, \quad\;\, \nu=\pi+\,\frac{1}{2}\,\arctan(4).
\end{align*}

Таким образом, в процессе оптимизации целевой функции должны использоваться методы, которые позволяют избежать попадания в точку локального минимума, например, метод отжига.

\section{Оптимизация}

%#############################################################################################################################################################################################################################
%################################################################### 2 Вариационный квантовый алгоритм на основе метода отжига ###############################################################################################
%#############################################################################################################################################################################################################################

\chapter{Вариационный квантовый алгоритм на основе метода отжига}

%#############################################################################################################################################################################################################################
%##################################################################################### 2.1 Метод отжига ######################################################################################################################
%#############################################################################################################################################################################################################################

\section{Метод отжига}

Метод отжига, как фундаментальная концепция в решении задач глобальной оптимизации, находит широкое применение в квантовых вычислениях, особенно в контексте вариационных квантовых алгоритмов. Основная идея метода заключается в постепенном снижении "температуры" системы, чтобы достичь состояния минимальной энергии. В этом разделе подробно рассмотрим как классический, так и квантовый подходы к отжигу, их теоретические основы и практическое применение.

Классический метод отжига основывается на аналогии с физическим процессом термического отжига, при котором материал медленно охлаждается, чтобы избежать образования дефектов и достичь состояния минимальной энергии. Математическое основание метода связано с распределением Больцмана, которое описывает вероятность состояния системы при заданной температуре $T$:

\begin{equation}
p(x) = \frac{1}{Z(T)} \exp\left(-\frac{E(x)}{k_B T}\right),
\end{equation}

где $E(x)$ — энергия состояния $x$, $k_B$ — константа Больцмана, а $Z(T)$ — статистическая сумма. Процесс отжига моделирует систему, которая может переходить между состояниями $x$ и $y$ с вероятностью, зависящей от разности энергий $\Delta E = E(y) - E(x)$:

\begin{equation}
p(x \rightarrow y) = \min\left(1, \exp\left(-\frac{\Delta E}{k_B T}\right)\right).
\end{equation}

С течением времени, температура $T$ постепенно уменьшается, что приводит к уменьшению вероятности перехода в состояния с более высокой энергией, в то время как система стремится к состоянию глобального минимума энергии.

Квантовый алгоритм отжига использует преимущества квантовой механики, такие как суперпозиция и туннелирование, для более эффективного поиска глобального минимума. В отличие от классического подхода, квантовый отжиг позволяет системе преодолевать энергетические барьеры, используя когерентное туннелирование, что значительно увеличивает вероятность нахождения глобального минимума.

Квантовый отжиг моделируется с помощью временного гамильтониана, который постепенно изменяется от начального состояния к целевому:

\begin{equation}
H(t) = (1 - s(t)) H_B + s(t) H_P,
\end{equation}

где $H_B$ — начальный гамильтониан, часто представляющий собой простую задачу, такую как сумма операторов Паули $X$, а $H_P$ — проблема-специфический гамильтониан. Функция $s(t)$, изменяющаяся от 0 до 1, управляет эволюцией системы от начального состояния к состоянию минимальной энергии.

Эволюция квантовой системы описывается уравнением Шрёдингера:

\begin{equation}
i \hbar \frac{\partial}{\partial t} \ket{\psi(t)} = H(t) \ket{\psi(t)},
\end{equation}

где $\hbar$ — приведённая постоянная Планка. Это уравнение описывает, как квантовая система изменяется со временем под воздействием изменяющегося гамильтониана.

Важным аспектом квантового отжига является адъективная эволюция системы, которая позволяет системе оставаться в состоянии минимальной энергии в течение всего процесса. Это достигается за счёт медленного изменения параметра $s(t)$ в соответствии с адъективным теоремой:

\begin{equation}
\frac{ds}{dt} \ll \frac{\Delta^2}{\hbar \left\lVert \frac{dH}{ds} \right\rVert},
\end{equation}

где $\Delta$ — энергетический разрыв между основным и первым возбужденными состояниями гамильтониана.

%#########################################################################################################################################################################################################################
%##################################################################################### 2.2 Алгоритм ######################################################################################################################
%#########################################################################################################################################################################################################################

\section{Алгоритм}

%#########################################################################################################################################################################################################################
%############################################################### 2.3 Сравнительные результаты тестирования ###############################################################################################################
%#########################################################################################################################################################################################################################

\section{Сравнительные результаты тестирования}

%#########################################################################################################################################################################################################################
%################################################################################################### Заключение ##########################################################################################################
%#########################################################################################################################################################################################################################

\addcontentsline{toc}{chapter}{\hspace{5.5mm} Заключение}
\chapter*{Заключение}


%#########################################################################################################################################################################################################################
%################################################################################################### Литература ##########################################################################################################
%#########################################################################################################################################################################################################################

\addcontentsline{toc}{chapter}{\hspace{5.5mm} Литература}
\begin{thebibliography}{99}
    \bibitem{Tsirulev2020}
    V. V. Nikonov, A. N. Tsirulev. \textit{Pauli basis formalism in quantum computations}. Volume 8, No 3, pp. 1 – 14, 2020.\\
    (\href{https:doi.org/10.26456/mmg/2020-831} {\textit{doi:10.26456/mmg/2020-831}})

    \bibitem{Preskill2018}
    J. Preskill. \textit{Quantum Computing in the NISQ era and beyond}. Quantum, vol. 2, p. 79, 2018.\\
    (\href{https://quantum-journal.org/papers/q-2018-08-06-79/}{\textit{quantum-journal:q-2018-08-06-79}})

    \bibitem{Cerezo2021}
    M. Cerezo, et al. \textit{Variational Quantum Algorithms}. Nature Reviews Physics, vol. 3, pp. 625-644, 2021.\\
    (\href{https://www.nature.com/articles/s42254-021-00348-9}{\textit{nature:42254-021-00348-9}})

    \bibitem{Peruzzo2014}
    A. Peruzzo, et al. \textit{A variational eigenvalue solver on a photonic quantum processor}. Nature Communications, vol. 5, p. 4213, 2014.\\
    (\href{https://www.nature.com/articles/ncomms5213}{\textit{nature:ncomms5213}})

    \bibitem{Farhi2014}
    E. Farhi, J. Goldstone, and S. Gutmann. \textit{A Quantum Approximate Optimization Algorithm}. arXiv preprint arXiv:1411.4028, 2014.\\
    (\href{https://arxiv.org/abs/1411.4028}{\textit{arXiv:1411.4028}})

    \bibitem{McClean2016}
    J. R. McClean, et al. \textit{The theory of variational hybrid quantum-classical algorithms}. New Journal of Physics, vol. 18, p. 023023, 2016.\\
    (\href{https://iopscience.iop.org/article/10.1088/1367-2630/18/2/023023}{\textit{iopscience:1367-2630-18-2-023023}})

    \bibitem{Kandala2017}
    A. Kandala, et al. \textit{Hardware-efficient variational quantum eigensolver for small molecules and quantum magnets}. Nature, vol. 549, pp. 242-246, 2017.\\
    (\href{https://www.nature.com/articles/nature23879}{\textit{nature:nature23879}})

    \bibitem{Harrow2009}
    A. W. Harrow, A. Hassidim, and S. Lloyd. \textit{Quantum algorithm for linear systems of equations}. Physical Review Letters, vol. 103, no. 15, p. 150502, 2009.\\
    (\href{https://journals.aps.org/prl/abstract/10.1103/PhysRevLett.103.150502}{\textit{aps:PhysRevLett.103.150502}})

    \bibitem{Biamonte2017}
    J. Biamonte, et al. \textit{Quantum machine learning}. Nature, vol. 549, pp. 195-202, 2017.\\
    (\href{https://www.nature.com/articles/nature23474}{\textit{nature:nature23474}})

    \bibitem{Lopatin}
    A. A. Lopatin. \textit{Квантовая механика и её приложения}. Санкт-Петербургский Государственный Университет.\\
    (\href{https://math.spbu.ru/user/gran/sb1/lopatin.pdf}{\textit{math.spbu:user/gran/sb1/lopatin}})

    \bibitem{Aspuru-Guzik2005}
    A. Aspuru-Guzik, A. D. Dutoi, P. J. Love, M. Head-Gordon. \textit{Simulated Quantum Computation of Molecular Energies}. Science, vol. 309, no. 5741, pp. 1704-1707, 2005.\\
    (\href{https://www.science.org/doi/10.1126/science.1113479}{\textit{science:1113479}})

    \bibitem{Schuld2015}
    M. Schuld, I. Sinayskiy, F. Petruccione. \textit{An introduction to quantum machine learning}. Contemporary Physics, vol. 56, no. 2, pp. 172-185, 2015.\\
    (\href{https://www.tandfonline.com/doi/abs/10.1080/00107514.2014.964942}{\textit{tandfonline:00107514.2014.964942}})

    \bibitem{Daskin2014}
    A. Daskin, S. Kais. \textit{Decomposition of unitary matrices for finding quantum circuits: Application to molecular Hamiltonians}. The Journal of Chemical Physics, vol. 141, no. 23, p. 234115, 2014.\\
    (\href{https://aip.scitation.org/doi/10.1063/1.4904315}{\textit{aip:1.4904315}})

    \bibitem{Romero2018}
    J. Romero, R. Babbush, J. R. McClean, C. Hempel, P. J. Love, A. Aspuru-Guzik. \textit{Strategies for quantum computing molecular energies using the unitary coupled cluster ansatz}. Quantum Science and Technology, vol. 4, no. 1, p. 014008, 2018.\\
    (\href{https://iopscience.iop.org/article/10.1088/2058-9565/aad3e4}{\textit{iopscience:2058-9565/aad3e4}})

    \bibitem{Havlicek2019}
    V. Havlicek, A. D. Córcoles, K. Temme, A. W. Harrow, A. Kandala, J. M. Chow, J. M. Gambetta. \textit{Supervised learning with quantum-enhanced feature spaces}. Nature, vol. 567, pp. 209-212, 2019.\\
    (\href{https://www.nature.com/articles/s41586-019-0980-2}{\textit{nature:s41586-019-0980-2}})

    \bibitem{Moll2018}
    N. Moll, P. Barkoutsos, L. Bishop, J. M. Chow, A. Cross, D. J. Egger, S. Filipp, A. Fuhrer, J. M. Gambetta, M. Ganzhorn, et al. \textit{Quantum optimization using variational algorithms on near-term quantum devices}. Quantum Science and Technology, vol. 3, no. 3, p. 030503, 2018.\\
    (\href{https://iopscience.iop.org/article/10.1088/2058-9565/aab822}{\textit{iopscience:2058-9565/aab822}})
\end{thebibliography}

%#########################################################################################################################################################################################################################
%################################################################################################### Приложение ##########################################################################################################
%#########################################################################################################################################################################################################################

\addcontentsline{toc}{chapter}{\hspace{5.5mm} Приложение  C{$\#$}}
\chapter*{Приложение C{$\#$}}

\end{document} 