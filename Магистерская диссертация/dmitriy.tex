\documentclass[a4paper]{report}

\def\baselinestretch{1.1}
\usepackage[14pt]{extsizes}
\usepackage[utf8]{inputenc}
\usepackage[russian]{babel}
\usepackage{indentfirst}
\usepackage{mathrsfs}

%%%%%%%%%%%%%%%%% Символы, графика %%%%%%%%%%%%%%%%%%%%%

\usepackage[T2A]{fontenc}
\usepackage{amsmath,amssymb,amsfonts,amsthm}
\newcommand\dsone{\mathds{H}}
\usepackage{graphicx}
\usepackage{color}
\usepackage[pdftex,colorlinks,linkcolor=blue,citecolor=blue]{hyperref}
\usepackage{pgfplots}
\usepackage{tikz}
\usepackage{array}
\newcolumntype{P}[1]{>{\centering\arraybackslash}p{#1}}

%%%%%%%%% Разметка страницы %%%%%%%%%

\bibliographystyle{plain}  % Change this to your preferred style
\renewcommand{\thetable}{\arabic{table}}
\usepackage{indentfirst}
\topmargin=-1.5cm %отступ сверху
\oddsidemargin=0.4cm %отступ слева (нечетные страницы)
\evensidemargin=0.4cm %(четные страницы)
\textwidth=16cm %ширина текста
\textheight=24cm
\tolerance=800
\parskip=1ex

\pagestyle{plain}

\usepackage{listings}

\definecolor{codegreen}{rgb}{0,0.6,0}
\definecolor{codegray}{rgb}{0.5,0.5,0.5}
\definecolor{codepurple}{rgb}{0.58,0,0.82}
\definecolor{backcolour}{rgb}{0.95,0.95,0.92}

\lstset{
    extendedchars=true,  % Corrected this line
    backgroundcolor=\color{backcolour},
    commentstyle=\color{codegreen},
    keywordstyle=\color{magenta},
    numberstyle=\tiny\color{codegray},
    stringstyle=\color{codepurple},
    breakatwhitespace=false,
    breaklines=true,
    captionpos=b,
    keepspaces=true,
    numbers=left,
    numbersep=3pt,
    showspaces=false,
    showstringspaces=false,
    showtabs=false,
    tabsize=1,
    basicstyle=\fontsize{10}{12}\selectfont\ttfamily
}

\begin{document}

\begin{titlepage}
    \begin{center}
        Министерство науки и высшего образования РФ\\
        ФГБОУ ВО «Тверской государственный университет»\\
        Математический факультет\\
        Направление 02.04.01 Математика и компьютерные науки\\
        Профиль <<Математическое и компьютерное моделирование>>
    \end{center}

    \vspace{2.5cm}
    \begin{center}
        {Магистерская диссертация }

        \vspace{1.0cm}
        \large{Вариационный квантовый алгоритм с оптимизацией методом отжига}

        \vspace{1.5cm}
    \end{center}

    \begin{flushright}
        \begin{minipage}{80mm}
            Автор:\\
            Алешин Дмитрий Алексеевич

            \vspace{1.0cm}
            Научный руководитель:\\
            д. ф.-м. н. Цирулёв А.Н.
        \end{minipage}
    \end{flushright}

    \vspace{1.6cm}
    \noindent Допущен к защите:\\
    Руководитель ООП:\\[1cm]
    \underline{\qquad \qquad \qquad \qquad \qquad }
    В.П. Цветков \\

    \vspace{2.3cm}
    \vspace{-0.7cm}
    \begin{center}
        Тверь, 2025
    \end{center}
\end{titlepage}

\setcounter{page}{2}

\tableofcontents
\newpage

\chapter*{Введение}

\chapter{Вариационные квантовые алгоритмы: общая схема}

\section{Базис Паули}

\section{Целевая функция и анзац}

\section{Общая схема алгоритма}

\section{Оптимизация}

\chapter{Вариационный квантовый алгоритм на основе метода отжига}

\section{Метод отжига}

\section{Алгоритм}

\section{Сравнительные результаты тестирования}

\chapter*{Заключение}

\chapter*{Список литературы}

\chapter*{Приложение {\huge C{$\#$}}}

\end{document} 