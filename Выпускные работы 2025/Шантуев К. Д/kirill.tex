\documentclass[a4paper]{report}

\def\baselinestretch{1.1}
\usepackage[14pt]{extsizes}
\usepackage[utf8]{inputenc}
\usepackage[russian]{babel}
\usepackage{indentfirst}
\usepackage{mathrsfs}
\usepackage{graphicx}
\usepackage{float}
\usepackage{wrapfig}

%%%%%%%%%%%%%%%%% Символы, графика %%%%%%%%%%%%%%%%%%%%%

\usepackage[T2A]{fontenc}
\usepackage{amsmath,amssymb,amsfonts,amsthm}
\usepackage{graphicx}
\usepackage{color}
\usepackage[pdftex,colorlinks,linkcolor=blue,citecolor=blue]{hyperref}
\usepackage{pgfplots}
\usepackage{tikz}

%%%%%%%%% Разметка страницы %%%%%%%%%

\usepackage{indentfirst}
\topmargin=-1.5cm %отступ сверху
\oddsidemargin=0.4cm %отступ слева (нечетные страницы)
\evensidemargin=0.4cm %(четные страницы)
\textwidth=16cm %ширина текста
\textheight=24cm
\tolerance=800
\parskip=1ex

\pagestyle{plain}

\usepackage{listings}

\definecolor{codegreen}{rgb}{0,0.6,0}
\definecolor{codegray}{rgb}{0.5,0.5,0.5}
\definecolor{codepurple}{rgb}{0.58,0,0.82}
\definecolor{backcolour}{rgb}{0.95,0.95,0.92}

\lstset{
    backgroundcolor=\color{backcolour},
    commentstyle=\color{codegreen},
    keywordstyle=\color{magenta},
    numberstyle=\tiny\color{codegray},
    stringstyle=\color{codepurple},
    breakatwhitespace=false,
    breaklines=true,
    captionpos=b,
    keepspaces=true,
    numbers=left,
    numbersep=3pt,
    showspaces=false,
    showstringspaces=false,
    showtabs=false,
    tabsize=1,
    basicstyle=\fontsize{10}{12}\selectfont\ttfamily
}

\begin{document}

\begin{titlepage}
	\begin{center}
		Министерство науки и высшего образования РФ\\
		ФГБОУ ВО «Тверской государственный университет»\\
		Математический факультет\\
		Направление 02.04.01 Математика и компьютерные науки\\
		Профиль <<Математическое и компьютерное моделирование>>	
	\end{center}
	
	\vspace{2.5cm}
	\begin{center}
	
		{МАГИСТЕРСКАЯ ДИССЕРТАЦИЯ}
		
		
		\vspace{1.0cm}
		\large{Математические модели малоразмерных систем кубитов с разреженными гамильтонианами}
		
		
		\vspace{1.0cm}
	\end{center}
	
	
	
	\begin{flushright}
		\begin{minipage}{80mm}
			Автор:\\
			Шантуев Кирилл Дмитриевич
			
			\vspace{1.0cm}
			Научный руководитель:\\
			д. ф.-м. н. Цирулёв А.Н.
			
		\end{minipage}
	\end{flushright}
	
	
	\vspace{1.3cm}
	\noindent Допущен к защите:\\
	Руководитель ООП:\\[1cm]
	\underline{\qquad \qquad \qquad \qquad \qquad }
    В.П. Цветков \\
	\vspace{2.3cm}
	
	
	
	\begin{center}
		Тверь 2025
	\end{center}
	
	\date{}
\end{titlepage}

\setcounter{page}{2}

\tableofcontents
\newpage

% Abstract
\addcontentsline{toc}{chapter}{\hspace{7mm} Введение}

\section*{Введение}

%####################################################
%################# Глава 1 ##########################
%####################################################

\chapter{Гильбертово пространство малоразмерной системы \\кубитов}

%####################################################

\section{Базисы и измерения}

Базисы и измерения являются основными концепциями в квантовой механике, которые помогают описывать поведение квантовых систем. Различные виды базисов определяют, как мы можем представлять состояния системы, а процесс измерения показывает, как эти состояния могут изменяться в результате взаимодействия с окружающим миром.

Базисом гильбертова пространства принято называть линейно независимых и образующих всё пространство векторов. Базисы в гильбертовом пространстве играют ключевую роль в квантовой механике и функциональном анализе. Гильбертово пространство - это абстрактное математическое пространство, в котором определены скалярные произведения, и оно может быть конечномерным или бесконечномерным. Это множество векторов, на котором определено скалярное произведение, удовлетворяющее аксиомам комплексности, линейности, симметричности и положительной определенности. Примерами могут послужить пространство квадратируемых функций ${L^2}$ и пространство конечномерных векторов. Формально, если ${{|e_i\rangle}}$ - это базис, то любой вектор ${|v\rangle}$ в гильбертовом пространстве можно представить как линейную комбинацию базисных векторов:
    \begin{equation}\label{}
    |v\rangle=\sum c_i |e_i\rangle,
    \nonumber
    \end{equation}
где ${c_i}$ - комплексные коэффициенты.

В квантовой механике состояние системы описывается вектором состояния (волновой функцией), который можно представить в любом из следующих базисов: собственные состояния вектора (важным примером является использование собственных состояний эрмитовых операторов как базиса для описания состояния системы) или суперпозиция состояний (квантовое состояние может быть выражено как суперпозиция состояний из выбранного базиса - ${|\psi\rangle = \sum_n c_n |e_n\rangle}$, где ${c_n}$ - комплексные коэффициенты).

\section{Гамильтонианы}

\section{Преобразование Йордана-Вигнера в\\ квантовой химии}

\section{Искусственные гамильтонианы для \\задач оптимизации}

%####################################################
%################# Глава 2 ##########################
%####################################################

\chapter{Две модели малоразмерных систем}

%####################################################

\section{Молекула водорода}

\section{Моделирование GHZ и W состояний}

\section{Классическое моделирование квантовых систем}

%####################################################

\newpage
\addcontentsline{toc}{chapter}{\hspace{7mm} Заключение}
\section*{Заключение}

В работе получены следующие основные результаты:

%####################################################

\newpage
\addcontentsline{toc}{chapter}{\hspace{7mm} Литература}

\begin{thebibliography}{99}

\bibitem{Tsirulev-Andre}
А.Н. Цирулёв, Э. Андре. \textit{Моделирование запутанных состояний в кластерах кубитов}. Физико-химические аспекты изучения кластеров, наноструктур и наноматериалов. Выпуск 14, c.\,342\,--\,349, 2022.\\
(\textit{https://physchemaspects.ru/2022/doi-10-26456-pcascnn-2022-14-342/})

\bibitem{Steeb-Hardy}
W. Steeb, Y. Hardy. \textit{Problems and solutions in quantum computing and quantum information}. World Scientific, 348 p., 2004.\\
(\textit{https://www.worldscientific.com/worldscibooks/10.1142/6077})

\bibitem{Tsirulev-Nikonov}
V.V. Nikonov, A.N. Tsirulev. \textit{Pauli basis formalism in quantum computations}. Mathematical Modelling and Geometry. Volume 8, No 3, pp.1-14, 2020.\\
(\textit{https://mmg.tversu.ru/images/publications/2020-831.pdf})

\end{thebibliography}

%####################################################

\newpage
\addcontentsline{toc}{chapter}{\hspace{7mm} Приложение}
\section*{Приложение}

\end{document} 