\documentclass[12pt,a4paper]{article}
\usepackage[utf8]{inputenc}
\usepackage[russian]{babel}

%%%%%%%%%%%%%%%%% Символы, графика %%%%%%%%%%%%%%%%%%%%%

\usepackage{amsmath,amssymb,amsfonts,amsthm}
\usepackage{bm}
\usepackage{graphicx}
\usepackage{color}
\usepackage[pdftex,colorlinks,linkcolor=blue,citecolor=blue]{hyperref}
\usepackage{tikz}
\usetikzlibrary{external, arrows.meta}
\usetikzlibrary{positioning, through, shapes, snakes}


%%%%%%%%% Разметка страницы %%%%%%%%%
\topmargin=-1.5cm %отступ сверху
\oddsidemargin=-0.4cm %отступ слева (нечетные страницы)
\evensidemargin=-0.4cm %(четные страницы)
\textwidth=16cm %ширина текста
\textheight=24cm


\usetikzlibrary{external}
\tikzexternalize
\tikzset{external/force remake=true}

\tikzset{every edge/.style= {draw=cyan, line width=1mm, -{Triangle[scale=0.5]}},
N/.style args= {#1/#2}{rectangle, rounded corners, fill=#1, text width=#2, font=\normalsize, align=center}, N/.default = cyan!30/7em}


\newcommand{\ket}[1] {\!\!\;\ensuremath{\left|#1\right\rangle}}
\newcommand{\bra}[1] {\!\!\:\ensuremath{\left\langle#1\right|\!\!\:}}
\newcommand{\ketbra}[2]{\!\!\:\ensuremath {\left|#1\right\rangle\!\:\!\!\left\langle#2\right|}}
\newcommand{\braket}[2]{\ensuremath {\!\!\:\left\langle#1\!\!\: \left|\!\!\!\;\right.#2\right\rangle\!\!\;}}

\begin{document}

\begin{figure}
\begin{center}
\begin{tikzpicture}[scale=1.2]
% Входное состояние     % minimum size=27mm
\node[rectangle, fill=cyan!50, rounded corners=1ex, draw, minimum width=25mm, minimum height=27mm, align=center] (A) at (0,0) {Входное\\ состояние $\vphantom{\underbrace{J}}$\\ ${\ket0^{{\scriptscriptstyle\otimes}{n}}\!=\! \ket{0\ldots0}}$\vspace{-1.5ex}\\
$\scriptstyle{\cdots\cdots\cdots\cdots\cdots\cdots\cdots\cdots\cdots}$\vspace{-1.0ex}\\
$n$ кубитов};
% Анзац U \vspace{-1ex}
\node[rectangle, fill=cyan!50, rounded corners=1ex, draw, minimum width=25mm, minimum height=27mm, align=center] (B) at (5,0) {Анзац $\hat{U}(\bm\theta)$:\\
новое состояние$\vphantom{\int^a_a}$\\
${\ket{u(\bm\theta)}= \hat{U}(\bm\theta)\ket0^{{\scriptscriptstyle\otimes}{n}}}, \vphantom{\underbrace{A}}$\\ ${\bm\theta=(\theta_1\ldots\theta_m)}$};
% Измерения и вычисление энергии
\node[rectangle, fill=yellow!50!red!50, rounded corners=1ex, draw, minimum width=25mm, minimum height=27mm, align=center] (C) at (10,0) {Измерения.\\
Вычисление энергии$\vphantom{\int^a_a}$\\
${E=\bra{u(\bm\theta)}\hat{H}\ket{u(\bm\theta)}}$};
% Узлы возвратной стрелки вниз
\node[circle, fill=violet, radius=1pt, inner sep=0cm,  draw] (D) at (13.0,0) {};
\node[circle, fill=violet, radius=1pt, inner sep=0cm,  draw] (E) at (13.0,-3.5) {};
% Условие завершения
\node[diamond, fill=red!50, align=center, aspect=1.7, draw] (F) at (10,-3.5) {Условие\\ завершения$\vphantom{\int^a}$\\ цикла};
% Изменение параметров
\node[rectangle, fill=red!50, rounded corners=1ex, draw, minimum width=27mm, minimum height=30mm, align=center] (G) at (5,-3.5) {Изменение$\vphantom{\int_a}$\\ параметров $\bm\theta$\\ классическим$\vphantom{\int^a_a}$\\ компьютером};
% Выход
\node[rectangle, draw] (J) at (10,-6.0) {\textbf{Результаты}};
\node  at (10.3,-5.0) {Да}; \node  at (7.5,-3.3) {Нет};
% Узлы возвратной стрелки вверх
\node[circle, fill=violet, radius=1pt, inner sep=0cm,  draw] (H) at (2.2,-3.5) {};
\node[circle, fill=violet, radius=1pt, inner sep=0cm,  draw] (I) at (2.2,0) {};
% Стрелки
\draw[line width=1pt, violet, tips, ->, -{Triangle}] (A) -- (B);
\draw[line width=1pt, violet, tips, ->, -{Triangle}] (B) -- (C);
\draw[line width=1pt, violet, tips, ->, -{Triangle}] (C) -- (D) -- (E) -- (F);
\draw[line width=1pt, violet, tips, ->, -{Triangle}] (F) -- (G);
\draw[line width=1pt, violet, tips, ->, -{Triangle}] (G) -- (H) -- (I);
\draw[line width=1pt, violet, tips, ->, -{Triangle}] (F) -- (J);
%\draw[line width=1pt, violet, tips, ->, -{Stealth}] (label3) -- (label4);
\end{tikzpicture}
\end{center}
\end{figure}


\end{document}
